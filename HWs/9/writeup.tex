\documentclass[11pt]{article}
\usepackage{amsmath,textcomp,amssymb,geometry,graphicx,enumerate}

\def\Name{Quoc Thai Nguyen Truong}  % Your name
\def\SID{24547327}  % Your student ID number
\def\Login{cs170-ig} % Your login (your class account, cs170-xy)
\def\Homework{9}%Number of Homework
\def\Session{Fall 2014}


\title{CS170--Fall 2014 --- Solutions to Homework \Homework}
\author{\Name, SID \SID, \texttt{\Login}}
\markboth{CS170--\Session\  Homework \Homework\ \Name}{CS170--\Session\ Homework \Homework\ \Name, \texttt{\Login}}
\pagestyle{myheadings}

\newenvironment{qparts}{\begin{enumerate}[{(}a{)}]}{\end{enumerate}}
\def\endproofmark{$\Box$}
\newenvironment{proof}{\par{\bf Proof}:}{\endproofmark\smallskip}

\textheight=9in
\textwidth=6.5in
\topmargin=-.75in
\oddsidemargin=0.25in
\evensidemargin=0.25in

\newcommand{\tab}{\hspace*{2em}}


\begin{document}
\maketitle

\noindent
Collaborators: God


\section*{DNA Subsequence}
\noindent
\textbf{Main idea.}\\
The main idea is to have a OverLap function which take 2 sub-strings and return a tuple of a overlap between 2 sub-strings and the maximum overlap of 2 sub-strings. Then, we read the text file and append it to the array (using close set so the array won't have duplicate sub-string).\\
There are one while loop (outer) and one for loop (inner), call OverLap function between s1, and s2, keep track of the maximum overlap which store a tuple of 3 elements (overlap string, number of maximum overlap, index of s2 in the array) . s1 is alway the 1st element in the array and s2 is the rest of the element in the lines array. If all the overlaps are NOT none (k[1] $\neq$ 0), then remove s1 (1st element in the array) and s2 from the array , and append the most overlap string between 2 sub-strings. If all the overlaps are none which we know from k (k[1] = 0 overlap), then add remove s1(1st element in the array) and append s1 to the array (end of the array)\\

Let M(i,j) = the string which has the maximum number of overlap of sub-string lines[i] and sub-string lines$[i+1,i+2,\cdots, j]$\\
\[ M(i,k) = Max \left\{ 
  \begin{array}{l l}
   OverLap(lines[i], lines[i+1])\\
   OverLap(lines[i], lines[i+2])\\
   \cdots\\
   OverLap(lines[i], lines[j])\\	
  \end{array} \right.\]
$$If\ M(i,j)\ is\ None\ ,\ remove\ lines[i]\ and\ append\ it\ to\ lines$$
$$Else:\ remove\ 2\ substring\ of\ max\ overlap\ and\ append\ the\ new\ sub-string\ to\ lines $$
\\
$$RESULT = M(i,j)_{i:= 0\ to\ n-1} ^{j:= 1\ to\ n} $$

\noindent
\textbf{Pseudocode.}\\
Line 0: OverLap(s1,s2):\\
Line 1:\tab If len(s1) $<$ len(s2): s1,s2 = s2,s1\\
Line 2:\tab If len(s2) == 0: return s1\\
Line 3:\tab k = ("",0,0) keep track of the most number of over lap between 2 substrings \\
Line 4:\tab Loop 1: If s2 is complete substring of s1, update k the most number of over lap \\
Line 5:\tab Loop 2: If s1 and s2 are overlap on the tail, update k the most number of over lap\\
Line 6:\tab Loop 3: If s2 and s2 are overlap on the head, update k the most number of over lap\\
Line 7:\tab If k[1] == 0: return None $\leftarrow$ there is no overlap between 2 sub-string\\
Line 8:\tab return (k[0],k[1]) $\leftarrow$ tuple (string, number of overlap)\\
\\
Line 9: lines = []\\
Line 10: lines = read all the lines from the text file and append each substring(using close set so the array won't have duplicate sub-string) \\
Line 11: While(len(lines) $>$ 2):\\
Line 12:\tab s1 = lines[0]\\
Line 13:\tab k = ("",0,0)\
Line 14:\tab for $i:= 1$ to len(lines):\\
Line 15:\tab S = OverLap(s1,line[i])\\
Line 16:\tab If S $\neq$ None:\\
Line 17:\tab\tab If k[1] $<$ S[1]:\\
Line 18:\tab\tab\tab update k = (S[0],S[1],i)\\
Line 18:\tab If (k[1] $\neq$ 0):\\
Line 19:\tab\tab remove element s1 and element at index k[2] from lines\\
Line 20:\tab\tab append the new overlap string k[0] to lines\\
Line 21:\tab Else: remove s1 (1st element) from lines and append s1 to lines\\
\\
Line 22: If len(lines) == 2: s1 = overlap(line[0], lines[1]), and output(s1[0])\\
Line 23: Else : output(lines[0])\\

\noindent
\textbf{Running time.}
$$\boxed{T(n,k) = \Theta(n^2k^2)}$$ 


\noindent
\textbf{Justification of running time.}\\
Let T(k,n) be the run time of the algorithm with n lines of substring at the most length of k.\\
From line 0 to line 8: It's the OverLap function, there are 3 different cases, and each case is the loop of checking between 2 substring which take $\Theta(k^2)$, and return the tuple of (string, number of overlap)\\
From line 11 to line 22: has 1 while loop and one for loop. the inner loop (for loop) call OverLap function between s1, and s2. s1 is alway the 1st element in the lines array and s2 is the rest of the element in the lines array. There are n of substrings, and we do this for every single substring. Hence, it will call Overlap function $\Theta(n^2) times$, and each time calling overlap take $\Theta(k^2)$. Therefore, the total run time would be:
$$\boxed{T(n,k) = \Theta(n^2k^2)}$$ 
\newpage


\end{document}
